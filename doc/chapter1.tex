
	\chapter{Introduction}
	
Neurodevelopmental disorders (NDD) are a group of disorders in which the development of the central nervous system is disturbed. This can include developmental brain dysfunction, which can manifest as neuropsychiatric problems or impaired motor function, learning, language or non-verbal communication \cite{Nature}.
It should be noted that autism falls under the definition of NDD and it is not a synonym.\\


\noindent Our project \textit{"SAM Activities"} consists in a set of activities targeted to support the study and the treatment process for patients affected by NDD. These activities take place in a controlled multi-sensory environment 
%cita paper magika%
equipped with a Kinect, SmartPlugs, SmartLights and the Dolphin SAM smart toy.
In particular our solution provides support to both the patients and the therapists in multiple ways.\\


\noindent From the patients’ needs point of view we developed two different kind of activities in order to stimulate them and to let them explore multiple level of interaction with the smart toy and the Magic Room:
\begin{description}
	\item [The Run activity] Which consists in a obstacle course game mode, during which the patient, interacting with the Dolphin SAM smart toy, is able to move around the virtual enviroment and overcome static and dynamic obstacles that will get in his/her way.
	\item [The Search activity] Which consists in a search\&find game mode, during which the patient, interacting with the Dolphin SAM smart toy and with the room space using the Kinect, is requested to search for seastars hidden within a virtual game area.
\end{description}
From the therapists needs' point of view our solution provides full support in the monitoring of the patients’ session. In fact the activities work with the support of a website via which the therapist can register, add his/her patients records with all the relevant information, and associate them the current session data in order to store them for later consulting. %in which the session informations will be stored and available for later consulting.
 Furthermore the therapist can customize, either from the website or directly from the game, the level of difficulty of the activities by setting specific parameters based on a particular patient abilities as he/she finds more suitable and store those settings for future runs.
	