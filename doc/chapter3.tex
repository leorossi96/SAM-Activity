\chapter{State of the Art}

Although multi-sensory environments and smart objects can be integrated in numerous different ways, the number of actual projects that mixes both concepts in a \textbf{balanced way} are very limited. 

Every related work that we are taking into account is either mainly (or totally) focused on the first aspect \cite{mora-guiard_lands_2016}\cite{ringland_sensorypaint:_2014}\cite{pares_promotion_2005}, or on the second one  \cite{yao_rope_2011}.

The concept of \textit{multimodality} in particular is seen by Birchfield \cite{birchfield_embodiment_2008}, together with the concepts of \textit{embodiment (physically exploration of concepts and systems by moving within and acting upon an environment) and composition(composition of new interaction scenarios, and extensibility of the base toolset)} as a major theme for building learning environments. 

Following these concepts, the Magic Room system \cite{garzotto_magic_2018} and its successor Magika \cite{gelsomini_magika_2019}, on which we based our work, represents one of the best examples of interactive learning environments. The system is in fact built connecting together multiple sensors, actuators, and output devices. 
The connection is taken in charge by a multilayered software platform that also orchestrate all the elements behaviors.

Other relevant projects are Sensory Paint \cite{ringland_sensorypaint:_2014} and Mediate \cite{pares_promotion_2005}.

\begin{description}
	\item[Mediate] A multi-sensory environment used to interact  with autistic children through a continuous generation of real-time stimuli. The interaction is based on action-reaction dialogues with the system, i.e. contingent interactions.
	The system is then built to \textbf{enhance non-repetitive behaviors}, since the repetition of typical attitudes are in general considered undesirable by psychologist, particularly referred to autistic children.
	
	The main idea was to \textbf{not give explicit elements of reaction}, but let the environment to react automatically, whatever action is taken by the user, creating \textit{the environment as a sort of living thing}. 
	The problem of typifying the various users is then presented, and solved through an interaction-driven design.
	
	\item[Sensory Paint] A multimodal Natural User Interface system, that enable users to paint with textured colored balls, showing a superimposed reflection of the user on a wall, to highlight the actions of the user. 
	It combine sensory integration and body awareness in a multisensory environment, i.e. a multimodal environment.
	
	The activities are presented and carried on in a task-oriented fashion, providing in this way engagement and sense of purpose.
	
	Overall the study also demonstrates that \textbf{multimodal systems can improve motor skills and body awareness as well as coordination}.
	As depicted in the evaluation, \textbf{the time of the interaction should not be too long}, since some participants found it not fun otherwise.
\end{description}

Additionally, some other works are really useful to define design principles for the creation of activities that uses certain devices. 
For example, the ones defined in the M4all project \cite{birchfield_embodiment_2008}, which consists of multiple Kinect-based activities developed for creating natural and playful learning experiences.
The activities are, in fact, built following multiple design principles:

\begin{itemize}
	\item Simple level geometry
	\item Slow pace
	\item Simple Level Flow
	\item Limited simultaneous actions
	\item Balanced required effort
	\item Non-precise timing 
	\item Limited consequences for errors
\end{itemize}

\noindent More smart-object-focused works explore the interaction between the user and the object, through sensors and actuators, which basically render the object a controller for the virtual activity. A relevant work on this aspect is Rope Revolution \cite{yao_rope_2011}.

\begin{description}
	\item[Rope Revolution]  A rope-based gaming system for collaborative play.
	The system is designed as a rope module with a motion sensing handle at the free end, and a force feedback mechanism, attached to a wall, to which the fixed end is mounted to, creating the \textbf{illusion that someone, or something, in the virtual world is pulling the rope}.
	Additionally, a projection of the virtual environment is then projected on the wall. Multiple users are connected through the internet in order to create a collaborative multiplayer environment.
	
	Both the gesture recognition and the force feedback systems are used in different ways, according to the game mode selected.
	There are a lot of different game modes: kite flying, rope jumping, horse driving, and wood sawing. Each of the activities can be performed either by a single player or multiple players remotely.
	
	The evaluation highlighted how the rope systems is very useful to enhance the integration between the physical and the virtual world, with some subjects highlighting how \textit{``it feels like the body was extended into the screen''}.
	Some other user pointed out that the rope interface provided visual feedback, not only physical, meaning that observing the movement of the rope help them also realize how to improve their movements.
	
	Finally, the system can assist creative play, and different combination of gestures that represent a variation from the standard behavior,  meaning that people are capable of generating new interaction methods based on the knowledge that they have of the system.
\end{description}

\noindent Finally, another relevant related work is Lands of Fog \cite{mora-guiard_lands_2016}, that represents also a partial example of how to integrate a smart object in those kind of environment as the main controller for the activity, although, in this case, the multisensory aspect is not exploited in a very deep fashion.

\begin{description}
	\item[Lands of Fog] A full-body-interaction system \cite{mora-guiard_lands_2016} designed to improve social and cognitive skills. 
	
	The physical materialization of the system is designed around a circular projection of six meters of diameter.
	Additionally, \textbf{the systems makes use of handheld physical pointers}, used by the children to interact with the underlining projection.
	This is due to the analysis of various psychologists that supported the idea that for ASD children interacting with a \textit{physical object can help in the mental mapping between the physical world  and the virtual one}.
	
	The virtual world is designed as \textbf{a natural environment covered by fog, of which only a small portion can be observed}.
	This practice is called \textit{``peephole"}, a design strategy which is proven to be good \cite{dalsgaard_between_2014} for improving exploration, since it motivates children to discover and explore the underlining world.
	This technique can also be useful for helping ASD children in focusing on just small samples of the environment, eliminating any other possible distraction.
	
	The world is the populated by fireflies, that can be captured, collected.
	At a certain point, multiple fireflies can transform into a bigger one, becoming a \textit{companion} for the child, \textit{creating a sense of empathy between the children and the creature}.
	Also, the creation of a persistent entity that follow the child can be useful to drive the child attention and behavior inside the game.
	
	The consequent evaluation demonstrate how the system can help the children in developing a certain \textit{mental flexibility and capacity of adapting to different game situations}, distancing from their repetitive behaviors. 
	Also the "peephole" technique result to be effective, since most children decrease the distance traveled across consecutive sessions, indicating a tendency of first understanding the world and then use that knowledge for a more efficient behavior. 
\end{description}

